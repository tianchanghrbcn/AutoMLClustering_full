\documentclass[10pt]{article} % 10pt 字体大小,单栏布局
%\documentclass[10pt,journal]{IEEEtran}
\usepackage{ctex}          % 支持中文
\usepackage{fontspec}      % 允许自定义英文字体(适用于 XeLaTeX 或 LuaLaTeX)
\usepackage{cite}
\usepackage{amsmath,amssymb,amsfonts}
\usepackage{graphicx}
\usepackage{booktabs}
\usepackage{geometry}
\usepackage{enumitem}
\usepackage{subcaption} % 添加子图/子表格支持
\usepackage{setspace}
\usepackage{titlesec}
\usepackage{tikz}
\usepackage{float}
\usetikzlibrary{shapes.geometric, arrows, positioning}
\usepackage[ruled,vlined]{algorithm2e} % 算法格式支持

% 页面布局设置
\geometry{a4paper, left=0.5in, right=0.5in, top=0.5in, bottom=1in} % 减小页边距

% 标题格式设置
\titleformat{\section}{\large\bfseries}{\thesection\ }{0em}{} % 一级标题无点
\titleformat{\subsection}{\normalsize\bfseries}{\thesubsection\ }{0em}{} % 二级标题无点
\titleformat{\subsubsection}{\normalsize\bfseries}{\thesubsubsection\ }{0em}{} % 三级标题无点

% 行距设置
\setstretch{1.2} % 行距 1.2 倍

% 公式编号统一格式
\numberwithin{equation}{section}

\begin{document}

\title{\textbf{数据清洗与下游聚类协同优化的自动化模型构建}}
\date{\today}
\maketitle

% 摘要部分
\noindent\textbf{\large 摘要} \\
在许多无监督学习任务中,现有的数据清洗与聚类技术已能在一定程度上降低噪声与缺失值带来的影响,但仍难以同时兼顾多样化清洗策略与聚类算法在大规模、高维数据中的协同需求。为进一步提升聚类质量与自动化效率,本文提出了一种清洗-聚类协同优化框架:通过多标签学习模型将数据的特征向量映射到最优或近优的“清洗-聚类”管线组合,从而在大幅减少搜索空间的同时确保较高的聚类性能。基于对 40 个公开数据集的大规模实验发现,不同数据特征会显著影响清洗与聚类的适配性,例如 Raha-Baran + HC 组合在高维、多特征数据上较为稳健,而 mode + DBSCAN 在低维数值数据上对噪声表现出极端敏感。通过该框架的自动化推荐与筛选,在部分场景下实现了平均 4.02 倍的搜索加速,并在保证聚类质量的同时取得了 -15.20\% 的平均损失率(即整体得分较基线有所提升)。研究结果表明,该方法对多样化数据具有一定的稳健性与可扩展性,为噪声较高、规模较大的真实数据环境提供了切实可行的聚类优化方案。未来工作将继续探索动态管线与流式数据下的协同优化策略,以更好地满足复杂实时场景的需求。

\section{引言}
在大数据与人工智能的快速发展背景下,无监督学习(如聚类分析)在医疗、金融及工业物联网等众多领域发挥着日益重要的作用\cite{10.3389/fncom.2019.00031, 7442571, app112311202}。例如,在医疗场景中,通过聚类可从病患数据中挖掘潜在群组,为个性化诊疗提供决策支持\cite{Sadeghi2021};在金融场景中,聚类方法可以帮助区分用户信用类别、增强客户预期回报的信心等\cite{Cai2016}。已有研究在聚类算法改进和可视化等方面取得了显著成果,常用的方法包括基于质心的 K-Means 及其变体\cite{Bandyapadhyay2024, Zhang2025}、基于密度的 DBSCAN\cite{Abdulhameed2024, Guo2024} 以及层次聚类\cite{Ran2023}、图聚类\cite{10.1145/3299876}等,这些方法为不同数据形态提供了有效的划分策略。与此同时,数据清洗技术(例如缺失值填补、异常值检测、错误值纠正)也在学术界与工业界获得广泛应用,用以降低噪声影响和提高数据质量\cite{10.1145/2723372.2749431, 10.1145/2463676.2465327, Rekatsinas2017} 。

然而,在无监督学习场景中,数据质量的影响往往更为突出。与分类或回归等有监督学习相比,聚类对于数据分布的依赖更强,一旦噪声、缺失值或错误值的比例较高,就可能破坏簇结构与真实分布之间的对应关系\cite{Atif2024, Sloutsky2012},从而对模式挖掘和决策支持造成不可忽视的干扰。虽然已有清洗方法在减少噪声方面效果显著,但过度或不当的清洗有时反而扭曲了关键特征\cite{Ni2023};此外,每种聚类算法对数据缺陷的敏感度并不相同\cite{SINGH2024102799},若只关注清洗或聚类单方面的优化,往往难以获得清洗策略与聚类算法之间的“全局最优”。

为解决上述问题,研究者逐步认识到“清洗策略 + 聚类算法 + 超参数”一体化管线的重要性\cite{Blumenberg2020, BOLANOSMARTINEZ2024102164}。这种做法能在保证数据分布尽量真实的同时,为不同数据集的特性“量体裁衣”地提供最佳策略。但由于管线搜索空间常呈指数级增长,且无监督场景缺乏显式标签指导,仅靠人工穷举或简单试验往往难以在可接受时间内完成参数寻优。近年来,自动化机器学习(AutoML)在有监督学习领域已呈现出显著优势,不仅能自动选择模型结构及超参数,还能优化特征工程\cite{SALEHIN202452}。然而,大部分 AutoML 研究集中于分类或回归任务,对无监督学习特别是“清洗+ 聚类”协同自动化的探索仍相对有限\cite{Bahmani2021ToTune, 9458702}。这为我们带来了新的机遇与挑战:能否将数据质量与无监督聚类的协同优化思路融入 AutoML 框架,从而在大规模及多场景下实现高效的自动管线搜索。

基于上述背景与需求,本文针对“数据清洗与下游聚类协同优化”这一交叉方向,提出了一种新的自动化管线优化模型。在该模型中,我们将多种数据清洗策略、聚类算法及其超参数统一纳入搜索空间,并借助多标签学习去构建“数据特征到优选方案”的映射关系:当面对一批新的数据集时,系统能快速推荐若干最优或近优的清洗-聚类-参数组合,在大幅缩减搜索规模的同时,保证聚类质量与效率。与传统手动调参或简单枚举策略相比,该方法在应对高维、噪声高或实时性强的场景时更具可行性。

\textbf{本研究的主要贡献包括:}
\begin{enumerate}
    \item \textbf{系统性地评估“清洗策略 × 聚类算法”组合的协同表现}

    基于 40 个具备多元质量问题的公开数据集,我们深入研究了不同噪声水平、错误率及规模下的三种清洗策略和六种聚类算法,对其组合在聚类质量、极端案例和时间开销方面进行了量化与比较。该评估不仅提供了对现有清洗-聚类方案适配度的系统认识,也为后续管线设计提供实用参考。
    \item \textbf{提出基于管线思维的协同优化框架}

    将“数据清洗 + 聚类 + 超参数”作为一个整体 Pipeline,并结合实验结果总结出多种针对性建议,帮助研究者在实际场景中有的放矢地进行策略选择,避免只在单一端的优化而忽略全局。
    \item \textbf{构建并验证了自动化管线优化模型}

    我们引入多标签学习来捕捉“数据特征与最优清洗-聚类组合”之间的关系,大幅减少了管线搜索空间。在对多个数据集的验证中,自动化模型在多数情况下能以3倍以上的加速比找到近似最优组合,并保持了较高的聚类准确度。该结果证明了将 AutoML 思路拓展到无监督学习领域的可行性与有效性。
    \item \textbf{为多样化数据场景的聚类优化提供可迁移路径}

    通过对损失率、加速比等指标的度量,我们量化了自动化管线在平衡质量与效率方面的潜力,为后续在工业领域部署这一思路打下基础,也为研究者进一步探索清洗与聚类协同优化的动态适配、在线更新等奠定了基础。

\end{enumerate}

我们的工作不仅加深了无监督学习场景下对数据清洗策略与聚类算法协同作用的理解,也为自动化机器学习在更广泛领域的落地探索了新路径。下文将依次介绍相关工作与研究背景(第 2 章)、问题与模型定义(第 3 章)、提出的方法与实现细节(第 4 章),并用大规模实验证明其有效性与适用性(第 5 章),最后对全文进行总结并展望未来研究方向(第 6 章)。


%---------------------------------
% 第二章:相关工作
%---------------------------------

\section{相关工作}

为更好地理解“清洗策略与聚类算法协同优化”在无监督学习场景下的研究现状,本文从三个方面回顾相关工作:第一,数据清洗与数据质量管理;第二,聚类算法及其改进;第三,自动化机器学习(AutoML)与无监督场景的探索。

\subsection{数据清洗与数据质量管理}
数据清洗旨在识别并修复各种数据缺陷(如缺失值、噪声、重复记录或错误值),是提升数据整体质量的重要途径\textcolor{blue}{\texttt{[待补参考文献:DataQualitySurvey]}}。已有研究在统计方法和机器学习方法方面均取得了丰硕成果。例如,早期工作主要依赖众数/均值填补\textcolor{blue}{\texttt{[待补参考文献]}}或规则驱动的异常值检测\textcolor{blue}{\texttt{[待补参考文献]}},在处理缺失值和简单错误时比较高效;后续研究则引入高级方法(如基于概率图模型或深度网络的自动修复\cite{Rekatsinas2017, 10.1145/2723372.2749431})以应对更复杂的错误类型。部分工作还引入了上下文约束或知识图谱\textcolor{blue}{\texttt{[待补参考文献:ContextualDataRepair]}},对特定领域(如医疗、经济数据)的不一致或罕见值进行更有针对性的纠正。

与此同时,研究者也认识到过度或不当清洗可能使原本有价值的异常点被误删或被扭曲\cite{Ni2023}。在有监督学习场景下,数据清洗常可借助标签对比来区分“真正有意义的异常”与“噪声性错误”\textcolor{blue}{\texttt{(补充点:引用相关工作)}};然而在无监督场景中,缺乏标签指导,清洗策略一旦过于保守或激进,就会对后续的聚类分析产生不可预测的影响\cite{app112311202}。这些研究进展表明,数据清洗方法的选择与配置应当与下游分析任务(如聚类)紧密结合,而非单独孤立地追求“最干净”的数据\textcolor{blue}{\texttt{[待补参考文献:CleaningAndAnalysisIntegration]}}。这也为我们随后探讨的“清洗与聚类协同优化”提供了重要动机。

\subsection{聚类算法及其改进}
聚类作为典型的无监督学习方法,已在图像识别、文本挖掘、用户分群等领域中得到了广泛应用。现有聚类算法大体可分为基于质心(如 K-Means 及其变体\cite{Bandyapadhyay2024, Zhang2025, IKOTUN2023178})、基于密度(如 DBSCAN、OPTICS\cite{Abdulhameed2024, Guo2024})与层次聚类\cite{HORNG2011306}三类。不同算法在簇形状、噪声耐受度、计算复杂度等方面各具优势\cite{SINGH2024102799}。

在面对不完美数据时,上述聚类算法往往对异常值和缺失值表现出不同的敏感度\cite{Atif2024, Sloutsky2012}。例如,少量异常点被 K-Means 视为远离中心的“噪声”,可在重新计算均值时抵消;但若这些点在 DBSCAN 的邻域定义中被错误识别,就可能导致过度分割\textcolor{blue}{\texttt{[待补参考文献:DBSCANNoiseImpact]}}。部分工作试图在算法内部引入鲁棒性机制,如改进距离度量或引入加权方案\textcolor{blue}{\texttt{[待补参考文献:RobustClusteringApproach]}},但大多仍需事先对数据进行相对独立的预处理。缺乏将“清洗策略”与“聚类算法”放在同一管线中统筹考量,往往导致在更复杂的高噪声场景中难以获得满意的聚类结果。

\subsection{AutoML 与无监督场景的探索}
近年出现的自动化机器学习(AutoML)框架,如 Auto-sklearn、TPOT、H2O AutoML 等\textcolor{blue}{\texttt{[待补参考文献:AutoMLSurvey, AutoSklearn]}},已在有监督学习(分类、回归)中展示了强大的自动化建模与调参能力。它们通常采用贝叶斯优化、强化学习或遗传算法等技术来搜索最优模型结构和超参数,以显著减轻人工调参成本\textcolor{blue}{\texttt{[待补参考文献:BayesianOpt, MetaLearningAutoML]}}。然而,大部分 AutoML 框架主要聚焦有监督任务,难以直接拓展到无监督领域。

已有少量研究注意到“AutoML + 无监督”的潜力,提出自动确定聚类数、自动选择聚类算法或自动化降维等方案\cite{Bahmani2021ToTune, 9458702},但在“清洗策略 + 聚类 + 参数”协同搜索方面尚未形成成熟体系\textcolor{blue}{\texttt{[待补参考文献:ClusterPipelineAutoML]}}。

\subsection{小结与差异}
综上所述,数据清洗与数据质量管理在过往研究中已经有了丰富的方法论与实用工具;聚类算法也有多种变体和改进,能适应不同数据分布与需求;AutoML 近年来极大简化了有监督学习场景下的模型选择与超参数调参。然而,三者在无监督场景下的“协同优化”尚缺乏系统化探索。

针对上述空缺,本文将立足于数据清洗与聚类的管线化思路,探索基于多标签学习方法的自动化管线搜索模型,力图在缩小搜索空间的同时保持或提升聚类质量。下文将先在第 3 章介绍问题定义与挑战,再在第 4 章提出具体的方法设计。


%---------------------------------
% 第三章:问题背景与挑战
%---------------------------------
\section{问题背景与挑战}\label{sec:problem-and-model}

\subsection{业务场景需求}\label{subsec:business-scenario}
在许多实际应用场景(例如医疗、金融以及工业物联网等领域),数据往往存在错误值(Error)、缺失值(Missing)以及噪声(Noise)等多种质量问题\cite{SantosArteaga2024,Zhou2020,Li2023,Lin2024}。这不仅影响了传统的数据分析流程,也对无监督聚类算法的结果稳定性和准确度造成了不利影响。当面对高维度或大规模的数据时,这种影响更为显著。一方面,数据清洗能够在一定程度上“修复”质量问题,以期保留更准确的分布结构;另一方面,聚类则希望在修复后的数据上挖掘有价值的簇结构,如在医疗数据中进行患者分群,在金融数据中挖掘客户风险等级,或在工业物联网中进行故障模式识别等。  
\vspace{0.3em}

随着业务复杂度的提升,数据清洗和聚类之间的结合需求愈发凸显:   
\begin{itemize}
    \item 在医疗场景下,高噪声或缺失的临床数据可能导致患者分群不准确,从而影响后续诊疗决策;
    \item 在金融场景中,信用数据的误填或异常值若未被及时修复,会干扰对客户分层和风险聚类的精准度;
    \item 在工业物联网中,传感器噪声与设备故障数据混杂,只有先行纠正错误值或剔除噪声,后续的聚类才更具可靠性。  
\end{itemize}
因此,“无监督聚类 + 数据清洗”的协同对实际业务的价值在于:通过精细化的数据质量保障,让聚类算法在有缺陷的数据上仍能产出更有意义和稳定的结果。

\subsection{技术需求与难点}\label{subsec:tech-challenges}
然而,实现上述业务目标并非易事。从技术层面看,主要面临以下\textbf{四大挑战}:
\begin{enumerate}
    \item \textbf{搜索空间大:} 
    不同的清洗策略(缺失值插补、异常值剔除、错误值纠正等),搭配多种聚类算法(如 K-Means、DBSCAN、层次聚类等)以及对应的超参数,会形成一个庞大的笛卡尔积。对大规模、高维数据而言,手动穷举或少量试验难以有效覆盖所有组合\cite{10346079}。

    \item \textbf{无监督评价困难:}
    与有监督学习不同,聚类没有明确标签来度量质量,需要依赖内部指标(如 Davie-Bouldin 指数、轮廓系数等)进行评估。这些指标对噪声和缺失值的分布较为敏感,也会放大清洗不当带来的偏差\cite{Atif2024,Sloutsky2012}。

    \item \textbf{噪声/缺失值处理的边界挑战:}
    少量噪声有时对 K-Means 等算法影响不大\cite{app112311202},但在基于密度的聚类(如 DBSCAN)中,某些“异常值”可能恰恰是重要模式的特征点。对高维、高错误率数据而言,清洗方式的选择与聚类算法的适配度至关重要\cite{Ni2023}。

    \item \textbf{自动化需求紧迫:}
    在实际场景中,数据规模与多样性日益增长,时间与人力往往无法支持逐一尝试各类清洗-聚类组合。如何实现一个自动化的端到端管线,既能减少冗余探索,又能保证结果质量,成为新的技术痛点\cite{Bahmani2021ToTune,9458702}.
\end{enumerate}

上述挑战表明,若想在高维、多样化数据中兼顾聚类质量与清洗效果,仅依赖单一的清洗或聚类算法已难以满足需求。反之,我们需要一个协同优化思路,将“数据清洗策略+聚类算法+超参数”视作统一管线,并借助自动化搜索与评估来平衡精度与效率。

\subsection{数学模型与形式化定义}\label{subsec:formal-definition}

为在理论与应用中更好地理解并解决“数据清洗与聚类算法”的协同优化,本小节对核心概念进行形式化定义,并建立相应的评价体系。

\subsubsection{核心概念与变量定义}
令 \(D\) 表示待处理的数据集,其中可能同时存在错误值(Error)、缺失值(Missing)以及噪声(Noise)等质量问题。为刻画这些问题与数据规模的差异,定义\textbf{特征向量}:
\begin{equation}\label{eq:xD}
  \mathbf{x}(D) 
  \;=\; 
  \Bigl(\mathrm{ErrorRate}(D),\;\mathrm{MissingRate}(D),\;\mathrm{NoiseRate}(D),\; m,\; n\Bigr),
\end{equation}
其中 \(\mathrm{ErrorRate}(D)\)、\(\mathrm{MissingRate}(D)\) 与 \(\mathrm{NoiseRate}(D)\) 分别表示数据集中错误值、缺失值以及噪声的相对比例,\(m\) 和 \(n\) 分别为数据的特征维度与样本规模。该向量不仅能在不同场景下进行对比,也为后续的自动化映射模型提供可学习的输入。

记 \(\mathcal{C}\) 为数据清洗方法的集合(如缺失值插补、异常值剔除、错误值纠正等),\(\mathcal{H}\) 为聚类算法集合(如 K-Means、DBSCAN、层次聚类等),\(\mathcal{P}\) 为聚类算法的超参数空间。将一个具体的清洗方法 \(c\)、聚类算法 \(h\) 及其超参数 \(\boldsymbol{\theta}\) 组合成\textbf{清洗-聚类策略}:
\begin{equation}\label{eq:omega}
  \omega 
  \;=\; 
  \bigl(c,\; h,\; \boldsymbol{\theta}\bigr),
\end{equation}
所有可行策略的笛卡尔积构成初始搜索空间:
\begin{equation}\label{eq:Omega}
  \Omega 
  \;=\; 
  \mathcal{C} \;\times\; \mathcal{H} \;\times\; \mathcal{P}.
\end{equation}
此时,如何在如此庞大的 \(\Omega\) 中高效找到适配度高的\(\omega\) 即是后续的研究重点。

\subsubsection{评价系统与最优方案}
为衡量任意策略 \(\omega \in \Omega\) 在数据集 \(D\) 上的聚类质量,通常采用若干\textbf{无监督评价指标}加以综合。本文主要使用\textbf{Davie-Bouldin(DB)指数}与\textbf{轮廓系数(Silhouette)}这两类典型指标,并线性组合为\textbf{综合得分}:
\begin{equation}\label{eq:S-score}
  S(D,\omega)
  \;=\;
  \alpha \cdot \bigl[-\,DB(D,\omega)\bigr]
  \;+\;
  \beta \cdot \mathrm{Sil}(D,\omega),
\end{equation}
其中 \(\alpha,\beta > 0\) 为可调权重,\(DB(\cdot)\) 越低表明簇内紧凑度与簇间分离度越理想,而 \(\mathrm{Sil}(\cdot)\) 越高代表类内相似度越高、类间差异越大\cite{Atif2024,Sloutsky2012}。若仅从聚类精度角度出发,给定数据集 \(D\) 的最优策略可表示为:
\begin{equation}\label{eq:best strategy}
  \omega^*(D)
  = \arg\max_{\omega \,\in\, \Omega} S(D,\omega).
\end{equation}
然而,若要在全量空间 \(\Omega\) 上评估每个策略 \(\omega\),往往需要极高时间成本。为此我们将定义一个\textbf{优化子空间} \(\Omega'(D)\subseteq \Omega\),仅在其中执行评估,以降低计算负担。若记评估单个策略的耗时为 \(T(D,\omega)\),则完整搜索与缩减搜索的总耗时分别为:
\begin{equation}\label{eq:T-original}
  T_{\text{original}}(D)
  \;=\;
  \sum_{\omega \,\in\, \Omega} \, T(D,\omega),
\quad
  T_{\text{reduced}}(D)
  \;=\;
  \sum_{\omega \,\in\, \Omega'(D)} \, T(D,\omega).
\end{equation}
通过一个合适的 \(\Omega'(D)\),希望在保证较高聚类质量的同时显著减少评估代价。

为量化“性能损失”与“时间加速”之间的平衡,引入\textbf{损失率}和\textbf{综合加速比}:
\begin{equation}\label{eq:loss-rate}
  \eta(D)
  =
  1 \;-\;
  \frac{\bar{S}(\Omega'(D))}{\bar{S}(\Omega)},
\end{equation}
\begin{equation}\label{eq:acc-ratio}
  \mathcal{A}(D)
  =
  \Bigl(1 - \eta(D)\Bigr)
  \;\times\;
  \frac{T_{\text{original}}(D)}{T_{\text{reduced}}(D)},
\end{equation}
其中 \(\bar{S}(\Omega)\) 表示在完整搜索上所得的平均得分,\(\bar{S}(\Omega'(D))\) 表示子空间\(\Omega'(D)\)上的平均得分,\(\eta(D)\) 越接近 0 表示缩减空间后带来的聚类性能损失越小,而 \(\mathcal{A}(D)\) 越大则表示更显著的加速效果。

\subsubsection{从数据特征到优选策略的映射}
在实际应用场景中,不同数据集 \(D\) 往往具有差异显著的\textbf{质量特征}(如 \(\mathrm{ErrorRate}\)、\(\mathrm{MissingRate}\)、\(\mathrm{NoiseRate}\) 等)。这些特征会显著影响清洗-聚类策略的效果,使得某些组合对特定类型数据更优。若能根据 \(\mathbf{x}(D)\)(参考式\eqref{eq:xD})提前预测哪些组合最可能获得高分,即可避免对整条搜索空间\(\Omega\)的全量评估。为此,我们引入一个映射函数:
\begin{equation}\label{eq:Omega-prime}
  G:\, \mathbf{x}(D)\,\mapsto\, \Omega'(D),
\end{equation}
其中 \(\Omega'(D)\subseteq \Omega\)。通过在先验数据集上学习“特征—策略表现”的关联,再在新数据集上借助该映射快速筛选候选方案,最终仅在子空间\(\Omega'(D)\)中执行评估即可极大降低时间成本。后续章节将介绍如何具体构建并训练这一映射。

\subsection{问题陈述}\label{subsec:problem-statement}
基于上述业务需求(第 \ref{subsec:business-scenario} 节)与技术难点(第 \ref{subsec:tech-challenges} 节),本研究的核心目标在于:\textbf{如何在可控的时间与资源约束下,找到对大规模/高噪声数据具有最优或近优效果的清洗-聚类方案?} 为便于形式化,本文聚焦以下三个关键子问题:

\begin{enumerate}[label=\textbf{(Q\arabic*)},leftmargin=15pt]
    \item \textbf{不同清洗-聚类组合在多种数据特征下的表现如何?}  
    如何定量比较并分析不同清洗方法和聚类算法在高维度、高错误率或缺失率等场景的效果差异,为选择最优或近优的策略提供参考依据。

    \item \textbf{如何基于数据特征构建“优选组合”映射函数?}  
    若数据集特征差异较大,仅采用统一的清洗或聚类策略并不理想。我们希望在先验数据上学习一个映射 \(G(\mathbf{x}(D))\mapsto \Omega'(D)\),在面对新数据集时自动筛选一批高潜力的组合,避免全量穷举。

    \item \textbf{如何平衡聚类质量与效率,实现在有限时间内逼近最优?}  
    在缩减搜索空间后,能否保证聚类质量的损失率 \(\eta(D)\) 可控甚至更优,并获得显著加速比 \(\mathcal{A}(D)\)?这对大规模多样化数据的实际应用尤为关键。
\end{enumerate}

围绕以上子问题,本文将在后续章节中进一步阐述自动化搜索与映射模型的设计思路,并通过实证实验验证其在多场景下的可行性与性能优势。


%---------------------------------
% 第四章:自动化聚类方法
%---------------------------------

\section{自动化聚类方法}
\label{sec:autoML}

为进一步提高清洗-聚类策略的搜索效率,本节将在第~\ref{sec:problem-and-model} 节所述概念的基础上,介绍将数据划分为先验数据与测试数据、使用多标签学习构建映射函数,以及最终实现自动化聚类优化流程的整体方法。该方法旨在通过离线阶段积累的先验知识,缩减在线搜索空间,从而在\textbf{较短时间}内找到\textbf{接近最优}的清洗-聚类组合并兼顾评估效率。
以下是本章节所定义的符号与描述:

\begin{table}[ht]
\centering
\small % 设置表格字体为0.8倍
\renewcommand{\arraystretch}{1.1} % 适当调整行距
\label{tab:symbols-advanced}
\begin{tabular}{ll}
\toprule
\textbf{符号} & \textbf{描述} \\
\midrule
$D_{\text{train}}$ & 先验数据集(训练集),用于离线评估和学习先验知识 \\
$D_{\text{test}}$ & 测试数据集,用于实际部署和快速优化 \\
$K$ & Top-K 大小,表示在先验阶段选取的前 $K$ 个最优方案 \\
$\mathbf{M}^{(i)}$ & 数据集 $D^{(i)}$ 的 Top-K 策略矩阵 \\
$\ell$ & 标签,表示某一优选方案的标识符 \\
$\mathcal{L}$ & 标签空间,包含所有优选方案的标签集合 \\
$\mathbf{L}^{(i)}$ & 数据集 $D^{(i)}$ 对应的多标签集合 \\
$\mathcal{M}$ & 训练集,包含所有先验数据的特征与标签集合 \\
$\mathcal{F}$ & 多标签分类器,用于预测优选方案标签 \\
$q^{(j)}$ & 标签 $\ell_{\omega^{(j)}}$ 为优选方案的概率 \\
$r$ & 预测阶段保留的最高优选标签数 \\
$\mathbf{L}'$ & 预测阶段保留的最高优选标签集合 \\
$\Omega'(D)$ & 数据集 $D$ 的优选子空间,$\Omega'(D) \subseteq \Omega$ \\
$G$ & 映射函数,将数据集特征向量映射到优选子空间 \\
$\hat{\omega}$ & 最优方案,即在 $\Omega'(D_{\text{test}})$ 中得分最高的组合 \\
\bottomrule
\end{tabular}
\caption{符号与描述}
\end{table}
\subsection{先验数据与多标签映射策略}
\label{sec:prior-data-mapping}


在实际应用中,通常可以从历史任务中获取大量已处理或部分标注的数据集,这些可视为\textbf{先验数据}(离线学习)。当面对新任务时,由于需要在较短时间内完成聚类策略的优选与评估,此时的新数据集则称为\textbf{测试数据}(在线应用)。通过在先验数据上深入探索并记录“数据特征—策略表现”的关联信息,就能在测试数据上显著减少不必要的搜索开销,从而提升整体效率。

\subsubsection{先验数据与测试数据的划分}
\label{subsec:dataset-split}

为便于在实际部署时利用先验知识,本研究将原有数据资源分为以下两类:
\begin{itemize}
    \item \textbf{先验数据集} $D_{\text{train}}$:由多个历史数据集组成,记为 ${D^{(1)}, D^{(2)}, \dots, D^{(N)}}$。在离线阶段(训练阶段),这些数据用于对搜索空间 $\Omega$ 进行大范围或抽样评估,以收集足够的策略得分信息,为后续自动化优化提供参考。
    \item \textbf{测试数据集} $D_{\text{test}}$:代表实际部署时面临的新数据,需要在线快速找到近优的清洗-聚类组合。此时可借助先验阶段所学知识,显著减少搜索规模并降低评估时间。
\end{itemize}

在离线评估过程中,若对每个先验数据集 $D^{(i)}$ 遍历或随机抽样若干清洗-聚类策略 $\omega \in \Omega$,便可计算各自方案的综合得分 $S(D^{(i)}, \omega)$。为高效记录在 $D^{(i)}$ 上表现最好的候选策略集,我们定义一个\textbf{Top-K 方案矩阵}(式~\eqref{eq:topK-matrix}),记为 $\mathbf{M}^{(i)}$,其中每一行是一个评分 $S_j$ 较高的策略组合 $\omega_j^{(i)}=(c_j,h_j,\boldsymbol{\theta}_j)$。该矩阵按照 $S_j$ 降序排列,用于在后续多标签学习中标识“优选”方案。

\begin{equation}\label{eq:topK-matrix}
\mathbf{M}^{(i)} 
= 
\begin{pmatrix}
c_1 & h_1 & \boldsymbol{\theta}_1 & S_1 \\
\vdots & \vdots & \vdots & \vdots \\
c_K & h_K & \boldsymbol{\theta}_K & S_K
\end{pmatrix}.
\end{equation}

\subsubsection{多标签学习与映射函数构建}
\label{subsec:multi-label}

在离线阶段,除了得到各数据集 $D^{(i)}$ 的 Top-K 策略外,还要提取其特征向量 $\mathbf{x}(D^{(i)})$。通过\textbf{多标签学习}的方法,可将“数据特征”与“优选策略集合”关联起来,从而在面对新数据集 $D_{\text{test}}$ 时,根据其特征向量 $\mathbf{x}(D_{\text{test}})$ 预测出最优或近优的策略子空间 $\Omega'(D_{\text{test}})$。

\paragraph{标签空间与多标签构造}  
在离线阶段,为了构建从数据特征到优选方案的映射模型,需引入\textbf{标签空间}的概念。首先,将所有先验数据集中出现过的“优选策略”记录下来,表示为:
\[
\{\omega^{(1)}, \omega^{(2)}, \ldots, \omega^{(m)}\},
\]
并为每个优选策略 $\omega^{(j)}$ 赋予唯一标签 $\ell_{\omega^{(j)}}$,从而形成一个离散的\textbf{标签空间}:
\begin{equation}\label{eq:label-space}
\mathcal{L}
= \{\ell_{\omega^{(1)}}, \ell_{\omega^{(2)}}, \ldots, \ell_{\omega^{(m)}}\}.
\end{equation}

对于某个先验数据集 $D^{(i)}$,其 Top-K 组合 $\mathbf{M}^{(i)}$(式~\eqref{eq:topK-matrix})中每个策略都可视为一个“正”标签。这些标签的集合定义为:
\begin{equation}\label{eq:label-space-for-D}
\mathbf{L}^{(i)}
= \{\ell_{\omega_1^{(i)}}, \ell_{\omega_2^{(i)}}, \ldots, \ell_{\omega_K^{(i)}}\}.
\end{equation}

结合数据集的特征向量 $\mathbf{x}(D^{(i)})$,可构造出多标签训练样本:
\[
\bigl(\mathbf{x}(D^{(i)}), \mathbf{L}^{(i)}\bigr).
\]
最终,所有先验数据集的多标签样本汇总成多标签训练集 $\mathcal{M}$:
\begin{equation}\label{eq:training-set}
\mathcal{M}
= \bigl\{\bigl(\mathbf{x}(D^{(1)}), \mathbf{L}^{(1)}\bigr), \ldots, \bigl(\mathbf{x}(D^{(N)}), \mathbf{L}^{(N)}\bigr)\bigr\}.
\end{equation}

\paragraph{分类器训练与映射生成}  
在完成多标签训练集 $\mathcal{M}$ 的构造后,下一步是利用该训练集对多标签分类器 $\mathcal{F}$ 进行训练。分类器的目标是学习数据特征 $\mathbf{x}(D)$ 与优选策略标签 $\mathcal{L}$ 之间的关联关系。

具体而言,分类器 $\mathcal{F}$ 的输出为每个标签 $\ell_{\omega^{(j)}}$ 的置信度 $q^{(j)} \in [0,1]$。对任意给定的新数据集 $D_{\text{test}}$,输入其特征向量 $\mathbf{x}(D_{\text{test}})$ 后,分类器将返回以下形式的预测结果:
\begin{equation}\label{eq:classifier}
\mathcal{F}\bigl(\mathbf{x}(D_{\text{test}})\bigr)
= \bigl\{(\ell_{\omega^{(1)}}, q^{(1)}), (\ell_{\omega^{(2)}}, q^{(2)}), \ldots, (\ell_{\omega^{(m)}}, q^{(m)})\bigr\},
\end{equation}
其中 $q^{(j)}$ 表示数据集 $D_{\text{test}}$ 在优选策略 $\omega^{(j)}$ 下的置信度。

为减少评估成本,仅选取置信度最高的 $r$ 个标签,构成优选标签集合:
\begin{equation}\label{eq:predicted-label-space}
\mathbf{L}' = \bigl\{\ell_{\omega^{(j)}} \,\mid\, q^{(j)} \text{ 属于前}r\text{大值}\bigr\}.
\end{equation}
将这些标签映射回对应的清洗-聚类策略,得到优化后的\textbf{优选子空间}:
\begin{equation}\label{eq:optimized-space}
\Omega'(D_{\text{test}})
= \bigl\{\omega^{(j)} \,\mid\, \ell_{\omega^{(j)}} \in \mathbf{L}'\bigr\}.
\end{equation}

此时,优选子空间 $\Omega'(D_{\text{test}})$ 通常远小于原始搜索空间 $\Omega$,从而在减少计算成本的同时,保持较高的聚类质量。最终,该映射过程可表示为:
\begin{equation}\label{eq:mapping-function}
G\bigl(\mathbf{x}(D)\bigr) = \Omega'(D).
\end{equation}

\subsection{自动化聚类优化流程}
\label{sec:autocluster-process}

在第~\ref{sec:prior-data-mapping} 节中,我们介绍了如何利用先验数据构建多标签映射策略,以学习数据特征 $\mathbf{x}(D)$ 与优选方案子空间 $\Omega'(D)$ 之间的映射关系。基于这一映射,本节将进一步探讨其在\textbf{自动化聚类优化流程}中的应用,重点分析如何利用该知识在新数据上高效筛选清洗-聚类组合,从而减少搜索空间并提高优化效率。

\begin{figure}[htbp]
  \centering
  \includegraphics[width=0.85\linewidth]{figures/autocluster_workflow.png}
  \caption{自动化聚类优化流程示意图}
  \label{fig:autocluster-workflow}
\end{figure}

如图~\ref{fig:autocluster-workflow} 所示,自动化优化流程主要包括\textbf{离线知识积累}(训练阶段)和\textbf{在线优化}(测试阶段)两个核心环节: \begin{enumerate} \item \textbf{训练阶段(离线学习)}:基于先验数据集 $D_{\text{train}}$,计算不同数据特征与清洗-聚类策略的匹配程度,并训练多标签分类器 $\mathcal{F}$,从而建立数据特征到优选方案子空间的映射 $G(\mathbf{x}(D))$。 \item \textbf{测试阶段(在线优化)}:面对新的数据集 $D_{\text{test}}$,利用训练阶段学习到的映射 $G(\mathbf{x}(D_{\text{test}}))$,快速筛选搜索空间 $\Omega$ 中的候选策略子集 $\Omega'(D_{\text{test}})$,避免全量穷举,从而在较短时间内获取高质量的清洗-聚类方案。 \end{enumerate}

在接下来的小节中,我们将详细介绍两个阶段的具体实现,并给出关键算法的伪代码。

\subsubsection{训练阶段:离线知识积累}
训练阶段的目标是基于先验数据集 \(D_{\text{train}}\) 生成多标签训练集并学习多标签分类器。算法伪代码如算法~\ref{alg:train-phase} 所示。

\begin{algorithm}[H]
\caption{离线训练阶段:生成训练数据与训练多标签分类器}
\label{alg:train-phase}
\KwIn{
    先验数据集 $D_{\text{train}}=\{D^{(1)},\dots,D^{(N)}\}$;\\
    搜索空间 $\Omega$;\\
    Top-K 大小 $K$。
}
\KwOut{多标签分类器 $\mathcal{F}$}

\SetKwFunction{GenerateTrainingData}{GenerateTrainingData}
\SetKwFunction{TrainClassifier}{TrainClassifier}

$\mathcal{M} \leftarrow \GenerateTrainingData(D_{\text{train}}, \Omega, K)$\;
$\mathcal{F} \leftarrow \TrainClassifier(\mathcal{M})$\;
\KwRet{$\mathcal{F}$}

\bigskip

\SetKwProg{Fn}{Function}{:}{}
\Fn{\GenerateTrainingData{$D_{\text{train}}, \Omega, K$}}{
  $\mathcal{M} \leftarrow \emptyset$\;
  \For{$i \leftarrow 1$ \KwTo $|D_{\text{train}}|$}{
    \ForEach{$\omega \in \Omega$ \textbf{(或采样自 $\Omega$)}}{
      计算 $S(D^{(i)}, \omega)$\;
    }
    选出 Top-K 策略 $\mathbf{M}^{(i)} = \{\omega_1^{(i)}, \dots, \omega_K^{(i)}\}$ 按得分降序\;
    映射为多标签集合 $\mathbf{L}^{(i)} = \{\ell_{\omega_1^{(i)}}, \dots, \ell_{\omega_K^{(i)}}\}$\;
    $\mathcal{M} \leftarrow \mathcal{M} \cup \{(\mathbf{x}(D^{(i)}), \mathbf{L}^{(i)})\}$\;
  }
  \KwRet{$\mathcal{M}$}
}

\Fn{\TrainClassifier{$\mathcal{M}$}}{
  \tcp{可根据具体多标签算法实现}
  训练多标签分类器 $\mathcal{F}$\;
  \KwRet{$\mathcal{F}$}
}
\end{algorithm}

\subsubsection{测试阶段:在线预测与最优方案搜索}
测试阶段在新数据集 \(D_{\text{test}}\) 上应用训练好的分类器,快速锁定优选子空间并搜索最优策略。伪代码如算法~\ref{alg:test-phase} 所示。

\begin{algorithm}[H]
\caption{测试阶段:寻找最优方案 \(\hat{\omega}\)}
\label{alg:test-phase}
\KwIn{
    测试数据集 $D_{\text{test}}$;\\
    多标签分类器 $\mathcal{F}$;\\
    搜索空间 $\Omega$;\\
    保留标签数 $r$。
}
\KwOut{最优方案 $\hat{\omega}$}

计算 $\mathbf{x}(D_{\text{test}})$\;
$\mathbf{L}' \leftarrow \{\}$\;
\ForEach{$\ell \in \mathcal{L}$}{
  $q_{\ell} \leftarrow \text{置信度}(\mathcal{F}, \mathbf{x}(D_{\text{test}}), \ell)$\;
  $\mathbf{L}' \leftarrow \mathbf{L}' \cup \{(\ell, q_{\ell})\}$\;
}
选取置信度最高的 $r$ 个标签 $\mathbf{L}'_{\mathrm{top}}$\;
映射回优选子空间 $\Omega'(D_{\text{test}})$\;
\ForEach{$\omega \in \Omega'(D_{\text{test}})$}{
    计算 $S(D_{\text{test}}, \omega)$ \tcp*{计算综合得分}
}
$\hat{\omega} \leftarrow \arg\max_{\omega \in \Omega'(D_{\text{test}})}S(D_{\text{test}}, \omega)$\;
\KwRet{$\hat{\omega}$}
\end{algorithm}

\subsection{小结}
本节系统介绍了自动化聚类方法的流程,通过离线学习和在线推断,实现了对大规模搜索空间的有效缩减,同时在保证精度的情况下显著提升了效率。

%---------------------------------
% 第五章:实验与结果分析
%---------------------------------

\section{实验与结果分析}
\label{sec:chapter5}

本章围绕第~\ref{sec:problem-and-model} 节所提出的问题和模型定义(特别是第~\ref{subsec:problem-formalization} 节)展开实验与结果分析。
我们将通过对多种数据集和聚类算法的验证,定量评估“数据清洗与聚类协同优化”方案的有效性和适用性,
并进一步检验基于自动化模型(多标签学习)的管线在实际场景中的性能表现。

\subsection{实验设置}
\label{sec:exp_setting}

本节将介绍实验所依赖的数据集、清洗策略与聚类算法等准备工作。

\subsubsection{数据集准备}
\label{sec:dataset_prep}

为确保实验覆盖多种数据质量问题(如缺失值、错误值、噪声等),我们选取了 40 个公开的数据集,这些数据集在规模、维度及数据缺陷的分布上存在显著差异。基于第~\ref{eq:xD} 节中对数据特征向量 $\mathbf{x}(D)$ 的定义,我们在实验前对每个数据集统计了错误率、缺失率和噪声率等关键指标。这些统计信息不仅有助于分析数据质量对聚类性能的影响,也为评估不同清洗-聚类策略在多种数据特征下的适配性提供了精细的对照依据。表~\ref{tab:datasets_info} 列出了部分数据集的关键统计信息。

\begin{table}[htbp]
    \centering
    \small % 设置表格字体为0.8倍
    \begin{tabular}{lcccccc}
    \toprule
    \textbf{数据集} & \textbf{数据集个数} & \textbf{样本数} & \textbf{特征数} & \textbf{缺失率范围} & \textbf{错误范围} & \textbf{平均IQR噪声率} \\
    \midrule
    flights  & 10  & 2376   & 7  & 0.00\% $\sim$ 49.99\% & 8.69\% $\sim$ 67.74\%  & 0.00\%  \\
    hospital & 10  & 1000   & 20 & 0.00\% $\sim$ 28.50\% & 8.53\% $\sim$ 49.29\%  & 2.62\%  \\
    beers    & 10  & 2410   & 11 & 4.04\% $\sim$ 31.82\% & 6.17\% $\sim$ 52.00\%  & 1.43\%  \\
    rayyan   & 10  & 1000   & 11 & 14.60\% $\sim$ 39.92\% & 10.75\% $\sim$ 52.73\% & 3.83\%  \\
    \bottomrule
    \end{tabular}
    \caption{实验中部分数据集的规模与质量问题概览}
    \label{tab:datasets_info}
\end{table}

\noindent
\vspace{-20pt} % 这里减少表格和下面内容的间距

\subsubsection{算法准备}
\label{sec:algo_prep}

本研究关注两方面算法:
(1) \textbf{数据清洗策略};(2) \textbf{聚类算法及对应参数}。

\paragraph{数据清洗策略}
我们在第~\ref{sec:related_work} 节介绍了若干种常用的数据清洗方法。本次实验选取了以下三种最具代表性的:
\begin{itemize}
	\item \textbf{Mode 填补}:该方法用来模拟基础的数据修复,主要通过众数或均值填补缺失值,并进行简单的错误值修正。它计算成本低,适用于数据缺失较少或错误较为简单的情况,但在面对复杂噪声时可能效果有限。
	\item \textbf{Raha-Baran}:作为深度数据修复的方法,该策略结合上下文规则与统计推断,能够识别并校正更复杂的错误和噪声。虽然能提升数据质量,但相较于基础填补方法,计算成本更高,适用于数据问题较综合且复杂的场景。
	\item \textbf{GroundTruth (GT)}:此方法仅用于对照实验,假设数据已被完全清洗,不含缺失值、错误值或噪声。它并非实际可行的清洗策略,而是用来评估其他数据修复方法的有效性。

\end{itemize}

\paragraph{聚类算法}
我们在实验中选取了 6 种经典且常用的无监督聚类算法,包括 \textit{K-Means}、\textit{DBSCAN}、\textit{OPTICS}、\textit{层次聚类 (HC)}、\textit{GMM} 以及 \textit{AffinityPropagation}。这些算法在聚类策略、超参数设置、计算复杂度及对噪声的敏感性等方面存在较大差异。为了更直观地比较它们的核心特性,表~\ref{tab:clustering_algorithms} 总结了每种算法的聚类方式、关键超参数、时间复杂度及参数敏感度。

\begin{table}[htbp]
    \centering
    \small % 设置表格字体为0.8倍
    \begin{tabular}{lcccc}
        \toprule
        \textbf{算法} & \textbf{聚类方式} & \textbf{算法参数} & \textbf{时间复杂度} & \textbf{参数敏感度} \\
        \midrule
        K-Means & 质心 & $k$(簇数) & $O(nkT)$ & 高 \\
        DBSCAN & 密度 & $\varepsilon$(邻域半径), MinPts(最小点数) & $O(n \log n)$ & 低 \\
        OPTICS & 密度 & MinPts(最小点数), $\xi$(相对密度变化阈值) & $O(n \log n)$ & 低 \\
        层次聚类 (HC) & 距离 & $k$(簇数), Linkage(链接方式), Metric(距离度量) & $O(n^2)$ & 高 \\
        GMM & 高斯混合模型 & $k$(高斯分布数), CovType(协方差类型) & $O(nkT)$ & 高 \\
        AffinityPropagation & 消息传递 & Damping(阻尼系数), Preference(偏好值) & $O(n^2)$ & 中 \\
        \bottomrule
    \end{tabular}
    \caption{所选聚类算法的特点比较}
    \label{tab:clustering_algorithms}
\end{table}

\noindent
从表~\ref{tab:clustering_algorithms} 可以看出,各聚类算法在适用场景、计算复杂度及对超参数的依赖程度上存在显著差异。例如,K-Means 计算效率较高,但对簇数选择较敏感,适用于球形簇结构的数据;DBSCAN 和 OPTICS 采用密度聚类方法,能够自动确定簇数,并对噪声具有较强的鲁棒性。层次聚类(HC)能够揭示数据的层次关系,但计算复杂度较高,不适合大规模数据。GMM 适用于数据呈高斯混合分布的情况,而 AffinityPropagation 则无需预设簇数,但计算开销较大,且受参数选择的影响显著。

在本实验中,我们将在不同数据清洗策略下评估这些聚类算法的表现,分析数据质量对聚类效果的影响,并比较各算法在不同数据特征下的适配性。

\vspace{1em}
\subsection{实验流程步骤}
\label{sec:exp_flow}

为保证实验的系统性与可复现性,我们设计了一套标准化的实验流程,如图~\ref{fig:exp_workflow} 所示。整体流程包括四个主要阶段:数据预处理、清洗与聚类执行、结果分析以及自动化聚类优化。以下是对实验流程图的详细解释:

\begin{figure}[htbp]
    \centering
    \includegraphics[width=1.0\linewidth]{figures/exp_workflow.png}
    \caption{实验设计流程示意图}
    \label{fig:exp_workflow}
\end{figure}

\begin{enumerate}
    \item \textbf{错误注入与特征统计}:  
    在原始数据集上人为注入不同比例的错误(包括缺失值、噪声和异常值),模拟真实数据质量问题。随后,统计各数据集的错误率、缺失率及噪声水平,提取数据特征向量 \(\mathbf{x}(D)\),为后续清洗与聚类优化提供基础记录。

    \item \textbf{清洗与聚类执行}:  
    依次应用不同的数据清洗策略,对每个数据集生成多个修复版本;随后,在清洗后的数据集上运行各种聚类算法,并记录对应的超参数、聚类簇数、运行时间及综合得分 \(S(D,\omega)\)(计算见式 \eqref{eq:S-score})。

    \item \textbf{结果分析}:  
    比较不同清洗-聚类组合的性能,评估其聚类质量、计算耗时及适用性,分析数据质量对聚类效果的影响,并总结最优策略的适配性。

    \item \textbf{自动化优化}:  
    在离线阶段训练多标签分类器,学习“数据特征 \(\mathbf{x}(D)\) → 优选子空间 \(\Omega'(D)\)”的映射关系;在测试阶段,该模型将用于高效推荐清洗-聚类方案,并与全量搜索结果进行对比,计算损失率 \(\eta(D)\) 和加速比 \(\mathcal{A}(D)\) 。
\end{enumerate}

该流程确保了实验在不同场景下的准确性和可复现性,并为后续实验提供统一框架。

在接下来的部分,我们将围绕两项核心任务展开:
\begin{itemize}
    \item \textbf{大规模对比实验(第~\ref{sec:large_scale_exp} 节)}:  
    评估各种清洗-聚类组合在全部数据集上的表现,分析最优策略在不同数据质量条件下的适配性。

    \item \textbf{自动化聚类模型评估实验(第~\ref{sec:automl_exp} 节)}:  
    验证自动化管线的有效性,对比多标签学习推荐的“优选子空间搜索”与“全量搜索”,重点考察损失率 \(\eta(D)\) 及加速比 \(\mathcal{A}(D)\)两个指标,评估其在大规模数据场景下的可行性。
\end{itemize}

\subsection{大规模对比实验}
\label{sec:large_scale_exp}

本节针对“清洗策略 + 聚类算法”协同优化进行大规模对比实验,以回答第~\ref{subsec:problem-formalization} 节提出的核心问题:
\emph{“不同清洗-聚类组合在多样化数据(含噪声、缺失值、错误值)上的协同表现如何?”}

\subsubsection{实验设置与评估指标}
\label{sec:exp_setting_largeset}

\paragraph{清洗-聚类组合}
对每个数据集分别应用 mode、Raha-Baran 及 GT 三种清洗方法,并运行 K-Means、DBSCAN、OPTICS、HC、GMM 和 AffinityPropagation 六种聚类算法,形成“3 清洗 × 6 聚类 = 18” 种组合。

\paragraph{评估指标}
本研究主要使用以下三个指标来评估清洗-聚类组合的效果:

\begin{itemize}
    \item \textbf{簇数量合理性}:  
    为确保聚类结果的有效性,簇数量应在以下范围内:  
    \begin{itemize}
        \item 标准范围:簇数量通常不小于 5 个,且不大于样本数的算术平方根。
        \item 筛选规则:若簇数明显偏离此范围(例如过少导致过度聚合,或过多导致过度分散),则视为不合法结果并排除。
    \end{itemize}
    
    \item \textbf{综合得分}:  
    综合得分 \(S(D,\omega)\) 的权重设置为 \(\alpha = 0.75\) 和 \(\beta = 0.25\),这一选择基于预实验结果。过高的轮廓系数权重会导致簇数过少,与实际需求不符。因此,较低的 \(\beta\) 权重有助于生成数量合适且更稳定的簇。

    \item \textbf{百分比得分}:  
    为了便于比较不同算法组合之间的性能,本研究对综合得分进行百分比归一化处理,以 GroundTruth 清洗策略修复后的最高综合得分为基准,将其定义为 100\%。不合理的实验结果(如算法运行超时、不收敛或簇数明显偏离标准范围)被标记为 0\%,以避免其干扰整体结果的分析。

\end{itemize}

\subsubsection{实验结果与分析}
\label{sec:exp_result_all}
本节围绕不同数据集上策略组合的性能展开实验结果呈现和分析。我们从以下三个层面展开讨论:
\paragraph{(1) 各数据集在不同错误率下的最优方案与得分情况}  
图~\ref{fig:beers_error}、\ref{fig:flights_error}、\ref{fig:rayyan_error} 和 \ref{fig:hospital_error} 分别展示了 \textit{beers}、\textit{flights}、\textit{rayyan} 与 \textit{hospital} 四个数据集在不同比例错误率(横轴)下的百分比综合得分(左侧纵轴)及缺失值比例(右侧纵轴)。我们引入了\textbf{最接近基准的组合},其得分是除参考标准外距离 100\% 最近的方案。该方案的选取有助于识别在实际应用中最接近理想基准性能的策略。

\begin{figure}[H]
    \centering
    \includegraphics[width=0.9\linewidth]{figures/legend_plot.png} % 确保 legend.png 存在
    \vspace{-14pt} % 减小图例与主图之间的间距
\end{figure}

\begin{figure}[H]
  \centering
  \begin{subfigure}{0.49\linewidth} % 每张图占 48% 宽度
    \centering
    \includegraphics[width=\linewidth]{figures/beers_error.png} % 替换为实际图片
    \caption{\textit{beers} 数据集:错误率 vs. 综合得分与缺失值比例}
    \label{fig:beers_error}
  \end{subfigure}
  \hfill
  \begin{subfigure}{0.49\linewidth}
    \centering
    \includegraphics[width=\linewidth]{figures/flights_error.png}
    \caption{\textit{flights} 数据集:错误率 vs. 综合得分与缺失值比例}
    \label{fig:flights_error}
  \end{subfigure}

  \vspace{0.5em} % 调整行间距

  \begin{subfigure}{0.49\linewidth}
    \centering
    \includegraphics[width=\linewidth]{figures/rayyan_error.png}
    \caption{\textit{rayyan} 数据集:错误率 vs. 综合得分与缺失值比例}
    \label{fig:rayyan_error}
  \end{subfigure}
  \hfill
  \begin{subfigure}{0.49\linewidth}
    \centering
    \includegraphics[width=\linewidth]{figures/hospital_error.png}
    \caption{\textit{hospital} 数据集:错误率 vs. 综合得分与缺失值比例}
    \label{fig:hospital_error}
  \end{subfigure}

  \caption{不同数据集在不同错误率条件下的得分与缺失值比例变化}
  \label{fig:all_datasets}
\end{figure}

\vspace{0.5em}
\noindent
通过以上四张图可以观察到,随着错误率的增加,各数据集的缺失值比例呈现不同程度的波动。此外,不同清洗-聚类组合在部分数据条件下表现出显著差异,其中某些方案的得分甚至远超基准值(超过 200\%)。为了更详细地分析这些现象,表~\ref{tab:beers_results} 至~\ref{tab:hospital_results} 进一步列出了四个数据集在不同错误率下的\textbf{最佳方案}(\textit{Best Combination}, C$_1$) 及其综合得分 (\textit{Best Score}, S$_1$),同时给出了\textbf{最接近基准的组合}(\textit{Best Deviation Combination}, C$_2$) 及其对应得分 (\textit{Best Deviation Score}, S$_2$)。这些结果有助于我们深入理解不同清洗-聚类组合的适用性,并评估其对聚类质量的影响。

\begin{table}[htbp]
    \centering
    \small % 设置表格字体为0.8倍
    \footnotesize % 适中字体
    \setlength{\tabcolsep}{4pt} % 调整列间距
    \renewcommand{\arraystretch}{1.1} % 增加行距,提高可读性
    \begin{subtable}{0.48\linewidth} % beers 数据集
        \centering
        \caption{\textit{beers} 数据集}
        \label{tab:beers_results}
        \begin{tabular}{lccccc}
            \toprule
            \textbf{Name} & \textbf{Error (\%)} & \textbf{C$_1$} & \textbf{S$_1$ (\%)} & \textbf{C$_2$} & \textbf{S$_2$ (\%)} \\
            \midrule
            beers & 6.17  & mode + HC  & 153.11 & R-B + HC  & 97.21 \\
            beers & 12.48 & R-B + HC   & 113.69 & mode + HC & 87.37 \\
            beers & 20.34 & mode + HC  & 259.85 & R-B + HC  & 134.50 \\
            beers & 26.60 & mode + HC  & 166.54 & mode + AP & 117.45 \\
            beers & 29.78 & R-B + HC   & 138.07 & mode + HC & 86.14 \\
            beers & 31.73 & mode + AP  & 232.95 & mode + DB & 49.49 \\
            beers & 40.23 & mode + HC  & 146.54 & mode + KM & 78.38 \\
            beers & 45.08 & mode + HC  & 208.15 & R-B + HC  & 80.19 \\
            beers & 46.73 & mode + HC  & 177.37 & mode + AP & 111.11 \\
            beers & 52.00 & mode + HC  & 127.06 & mode + DB & 102.51 \\
            \bottomrule
        \end{tabular}
    \end{subtable}
    \hfill
    \begin{subtable}{0.48\linewidth} % flights 数据集
        \centering
        \caption{\textit{flights} 数据集}
        \label{tab:flights_results}
        \begin{tabular}{lccccc}
            \toprule
            \textbf{Name} & \textbf{Error (\%)} & \textbf{C$_1$} & \textbf{S$_1$ (\%)} & \textbf{C$_2$} & \textbf{S$_2$ (\%)} \\
            \midrule
            flights & 8.69  & mode + HC  & 290.44 & mode + KM  & 105.36 \\
            flights & 16.97 & mode + GMM & 152.61 & R-B + HC   & 84.89 \\
            flights & 24.56 & R-B + HC   & 117.03 & R-B + AP   & 100.53 \\
            flights & 30.63 & mode + DB  & 242.21 & mode + GMM & 103.68 \\
            flights & 38.92 & mode + DB  & 1817.80 & mode + GMM & 103.55 \\
            flights & 40.76 & mode + DB  & 2543.51 & mode + AP  & 95.80 \\
            flights & 45.44 & mode + AP  & 130.95 & mode + DB  & 98.48 \\
            flights & 51.59 & mode + HC  & 135.66 & R-B + HC   & 109.68 \\
            flights & 62.87 & mode + KM  & 187.77 & R-B + HC   & 113.10 \\
            flights & 67.74 & mode + HC  & 204.94 & R-B + HC   & 104.27 \\
            \bottomrule
        \end{tabular}
    \end{subtable}

    \vspace{0.5em} % 适当增加表格间的垂直间距

    \begin{subtable}{0.48\linewidth} % rayyan 数据集
        \centering
        \caption{\textit{rayyan} 数据集}
        \label{tab:rayyan_results}
        \begin{tabular}{lccccc}
            \toprule
            \textbf{Name} & \textbf{Error (\%)} & \textbf{C$_1$} & \textbf{S$_1$ (\%)} & \textbf{C$_2$} & \textbf{S$_2$ (\%)} \\
            \midrule
            rayyan & 10.75  & R-B + HC  & 91.38  & R-B + HC  & 91.38  \\
            rayyan & 13.79  & R-B + HC  & 74.35  & R-B + HC  & 74.35  \\
            rayyan & 16.88  & R-B + HC  & 83.63  & R-B + HC  & 83.63  \\
            rayyan & 19.71  & mode + HC & 71.69  & mode + HC & 71.69  \\
            rayyan & 22.77  & R-B + HC  & 79.87  & R-B + HC  & 79.87  \\
            rayyan & 24.35  & R-B + HC  & 101.71 & R-B + HC  & 101.71 \\
            rayyan & 29.25  & R-B + AP  & 99.49  & R-B + AP  & 99.49  \\
            rayyan & 40.24  & R-B + HC  & 43.52  & R-B + HC  & 43.52  \\
            rayyan & 47.88  & mode + HC & 25.22  & mode + HC & 25.22  \\
            rayyan & 52.73  & mode + HC & 51.75  & mode + HC & 51.75  \\
            \bottomrule
        \end{tabular}
    \end{subtable}
    \hfill
    \begin{subtable}{0.48\linewidth} % hospital 数据集
        \centering
        \caption{\textit{hospital} 数据集}
        \label{tab:hospital_results}
        \begin{tabular}{lccccc}
            \toprule
            \textbf{Name} & \textbf{Error (\%)} & \textbf{C$_1$} & \textbf{S$_1$ (\%)} & \textbf{C$_2$} & \textbf{S$_2$ (\%)} \\
            \midrule
            hospital & 8.53  & R-B + HC  & 87.63  & R-B + HC  & 87.63  \\
            hospital & 11.96 & R-B + HC  & 77.42  & R-B + HC  & 77.42  \\
            hospital & 15.34 & R-B + HC  & 97.14  & R-B + HC  & 97.14  \\
            hospital & 21.65 & R-B + HC  & 69.24  & R-B + HC  & 69.24  \\
            hospital & 24.83 & R-B + HC  & 72.12  & R-B + HC  & 72.12  \\
            hospital & 27.96 & R-B + HC  & 77.16  & R-B + HC  & 77.16  \\
            hospital & 33.68 & R-B + HC  & 72.50  & R-B + HC  & 72.50  \\
            hospital & 36.52 & mode + AP & 66.43  & mode + AP & 66.43  \\
            hospital & 46.52 & mode + HC & 70.70  & mode + HC & 70.70  \\
            hospital & 49.29 & mode + AP & 92.17  & mode + AP & 92.17  \\
            \bottomrule
        \end{tabular}
    \end{subtable}

    \caption{不同错误率下各数据集的最佳组合 (C$_1$) 与最接近基准组合 (C$_2$) 及得分}
    \label{tab:all_results}
\end{table}

\vspace{0.5em}
\noindent
基于图~\ref{fig:beers_error} 至图\ref{fig:hospital_error} 和表~\ref{tab:beers_results} 至表~\ref{tab:hospital_results} 所示的不同错误率下的组合排名与偏差情况,我们可以得出以下几点认识:

\begin{enumerate}
    \item \textbf{“最佳组合”与“最接近基准”经常不一致}\\
    在 \textit{Beers}、\textit{Flights} 等较大规模、数值型为主的数据中,mode + HC 或 mode + DBSCAN 有时会出现 200\%~300\%、甚至逾千的“爆分”情形。然而,这些高分结果往往偏离基准结构较远,原因是 DBSCAN 对噪声/缺失值极度敏感,以及 HC 采用的簇数与理想情况差异较大。此时,Raha-Baran + HC 或 mode + KMeans 虽绝对分值较低,却更贴近 GroundTruth 基准。

    \item \textbf{极端高分分布并不均衡,受数据维度与语义约束影响}\\
    在低维数值型的 \textit{Flights} 数据中,mode + DBSCAN 出现 1800\%~2500\% 极端情况,表明密度聚类对填补策略和超参数高度敏感。而在更高维或包含语义规则的 \textit{Hospital}、\textit{Rayyan} 中,聚类分数通常较温和,极少越过 200\%,暗示 Raha-Baran 修复在此类场景下更能保留合理的分布结构,降低了极端划分的可能。

    \item \textbf{Raha-Baran + HC 在中低错误率下尤为稳健}\\
    在 \textit{Hospital}、\textit{Rayyan} 等数据集中,当错误率低于 25\% 时,Raha-Baran + HC 常同时取得“最高分”与“最贴近基准”的好结果。对于高维、具备一定语义约束且规模中等的场景而言,Raha-Baran 能有效减少噪声干扰,HC 则在数据质量良好时更易逼近基准结构。即使错误率升至 40\% 以上,Raha-Baran + HC 也仅出现有限波动,在高维场景下仍优于其他组合。
\end{enumerate}

\paragraph{(2) 不同数据集上清洗-聚类方案随错误率变化的趋势}
前文表格展示了各数据集在不同错误率下的最佳组合及其与基准的偏差,但仅反映离散的关键数值。为直观观察清洗-聚类组合随错误率变化的趋势,本节通过图~\ref{fig:mode_beers} 至图\ref{fig:raha_baran_hospital} 展示各数据集在两种清洗方法下,不同聚类算法的综合得分变化。横轴为错误率,纵轴为综合得分,各曲线对应不同聚类算法,体现其在不同数据质量条件下的波动情况。
\begin{figure}[H]
    \centering
    \includegraphics[width=0.6\linewidth]{figures/legend.png} % 确保 legend.png 存在
    \vspace{-10pt} % 减小图例与主图之间的间距
\end{figure}

\begin{figure}[H]
  \centering
  \footnotesize % 控制整体字体大小(子图内文本)
  \setlength{\abovecaptionskip}{2pt} % 调整标题和图片之间的间距
  \setlength{\belowcaptionskip}{0pt} % 让 caption 和正文的间距更小

  % 第一行:mode 清洗
  \begin{subfigure}{0.48\linewidth} % 图片尺寸增大
    \centering
    \includegraphics[width=\linewidth]{figures/mode_beers_combined_scores.png}
    \caption{\textit{mode}, Beers}
    \label{fig:mode_beers}
  \end{subfigure}
  \hfill
  \begin{subfigure}{0.48\linewidth}
    \centering
    \includegraphics[width=\linewidth]{figures/mode_flights_combined_scores.png}
    \caption{\textit{mode}, Flights}
    \label{fig:mode_flights}
  \end{subfigure}

  \vspace{0.2em} % 缩小两行之间的间距

  \begin{subfigure}{0.48\linewidth}
    \centering
    \includegraphics[width=\linewidth]{figures/mode_rayyan_combined_scores.png}
    \caption{\textit{mode}, Rayyan}
    \label{fig:mode_rayyan}
  \end{subfigure}
  \hfill
  \begin{subfigure}{0.48\linewidth}
    \centering
    \includegraphics[width=\linewidth]{figures/mode_hospital_combined_scores.png}
    \caption{\textit{mode}, Hospital}
    \label{fig:mode_hospital}
  \end{subfigure}

  \vspace{0.2em} % 适当减少两行之间的间距

  % 第二行:Raha-Baran 清洗
  \begin{subfigure}{0.48\linewidth}
    \centering
    \includegraphics[width=\linewidth]{figures/raha-baran_beers_combined_scores.png}
    \caption{\textit{Raha-Baran}, Beers}
    \label{fig:raha_baran_beers}
  \end{subfigure}
  \hfill
  \begin{subfigure}{0.48\linewidth}
    \centering
    \includegraphics[width=\linewidth]{figures/raha-baran_flights_combined_scores.png}
    \caption{\textit{Raha-Baran}, Flights}
    \label{fig:raha_baran_flights}
  \end{subfigure}

  \vspace{0.2em} % 继续缩小行距

  \begin{subfigure}{0.48\linewidth}
    \centering
    \includegraphics[width=\linewidth]{figures/raha-baran_rayyan_combined_scores.png}
    \caption{\textit{Raha-Baran}, Rayyan}
    \label{fig:raha_baran_rayyan}
  \end{subfigure}
  \hfill
  \begin{subfigure}{0.48\linewidth}
    \centering
    \includegraphics[width=\linewidth]{figures/raha-baran_hospital_combined_scores.png}
    \caption{\textit{Raha-Baran}, Hospital}
    \label{fig:raha_baran_hospital}
  \end{subfigure}

  \caption{不同数据集在不同错误率条件下的聚类算法综合得分变化趋势}
  \label{fig:all_combined_scores}
\end{figure}

\noindent

通过综合比对上述不同数据集在 mode 与 Raha-Baran 清洗下,各聚类算法随错误率变化的表现,可以得到以下几条规律性结论:

\begin{enumerate}
    \item \textbf{错误率升高易出现“极端爆分”或“收敛失败”,精细化清洗可缓解但不能完全避免} \\
    当错误率中高时,mode 清洗下的 HC、DBSCAN、AP 更易出现高达 3.00 的极端得分或直接收敛到 0.00。使用 Raha-Baran 虽然减少了此类“爆分”现象,但在个别极端场景下,KMeans、GMM 或 AP 仍可能无法正常聚类,导致分数骤降至 0.0。

    \item \textbf{聚类算法对错误率的敏感度差异明显,HC/DBSCAN 波动大,KMeans/GMM/OPTICS 相对平稳} \\
    层次聚类(HC)与密度聚类(DBSCAN)对噪声和超参数敏感度最高,最易在错误率升高后出现大幅波动或极端值。相比之下,KMeans 与 GMM 在大多错误率区间分数处于中等水平,不常爆分但也可能在高错误率时收敛失败;OPTICS 波动相对较小,但平均得分不高。

    \item \textbf{“mode” 与 “Raha-Baran” 的差异在高错误率下尤为突出} \\
    低或中等错误率下,两种清洗方法整体表现差异不大;但一旦错误率升至 30\% 以上,mode 更易触发少数算法的极端波动或收敛异常,而 Raha-Baran 虽也会出现偶发失效,却在多数场景保持了中等或较平稳的分数,展现出对高错误率的相对鲁棒性。
\end{enumerate}

\noindent
综上所述,随着错误率的持续升高,数据质量对聚类算法的影响呈显著放大趋势。尽管使用更精细的清洗方法(如 Raha-Baran)能够降低极端结果出现的概率,但在高噪声场景下仍有可能出现算法收敛失败。为更全面地评估不同清洗-聚类组合在各种数据集上的整体性能,我们将进一步从更多指标对策略的表现进行综合比较。

\paragraph{(3) 基于多指标的综合性能分析}
图~\ref{fig:alg_comb_metrics} 以 5 张柱状图展示了 18 种清洗-聚类组合策略在不同聚类评估指标下的整体表现,横轴为算法组合,纵轴为相应指标数值,评估指标包括:
\begin{itemize}
    \item \textbf{Average Score (\%)}:算法组合在所有实验场景中的平均得分百分比(0\%--150\%),反映其总体性能;
    \item \textbf{Average Combined Score}:综合分数均值(0--2),度量簇的紧致性与分离度的平衡;
    \item \textbf{Standard Deviation of Percentage Score}:得分百分比的标准差,表示不同场景间的波动大小;
    \item \textbf{Standard Deviation of Combined Score}:综合分数的标准差,用于评估算法在多场景下的一致性;
    \item \textbf{Average Deviation from Reference (100\%)}:相对于基准方案(100\%)的平均偏离程度,数值越小表示越接近基准。
\end{itemize}

此外,图~\ref{fig:alg_comb_metrics_low} 在上述基础上选取错误率低于 25\% 的数据,聚焦分析非极端噪声条件下各组合的聚类性能与稳定性,直观对比各方法在低错误率场景下的差异。以下是从这两类图表中得出的主要结论与启示:

\begin{figure}[htbp]
    \centering
    \setlength{\abovecaptionskip}{5pt}  % 增加标题与图片之间的间距
    \setlength{\belowcaptionskip}{5pt}  % 增加标题与正文之间的间距

    % 第一张图(单独占一行)
    \begin{subfigure}{1.0\linewidth}  % 增大宽度
        \centering
        \includegraphics[width=\linewidth]{figures/alg_comb_metrics.png}
        \caption{18 种清洗-聚类组合在不同聚类评估指标上的综合表现}
        \label{fig:alg_comb_metrics}
    \end{subfigure}

    \vspace{1em}  % 增加两张图之间的间距

    % 第二张图(单独占一行)
    \begin{subfigure}{1.0\linewidth}  % 增大宽度
        \centering
        \includegraphics[width=\linewidth]{figures/alg_comb_metrics_low.png}
        \caption{错误率低于 25\% 条件下,各算法组合的聚类性能与稳定性}
        \label{fig:alg_comb_metrics_low}
    \end{subfigure}

    \caption{聚类算法组合在不同条件下的评估指标比较}
    \label{fig:alg_comb_metrics_comparison}
\end{figure}
\vspace{-10pt}

\subparagraph{全局方案表现与离散度分析}
图~\ref{fig:alg_comb_metrics} 将所有数据集(含不同错误率)的结果归纳,便于从全局角度评估各方案的稳定性与适应性。可以看到:
\begin{itemize}
    \item \textbf{高均值+高方差的“爆分”组合}:例如 mode + DBSCAN 组合虽然平均分高达 147\%,却伴随多达 481.75 的标准差 ,说明其在数值特征为主、规模较大的数据(如 \textit{Flights, Beers})上易发生极端聚类效果,在实际使用中需谨慎。
    \item \textbf{Raha-Baran + HC 和 mode + HC 整体分数高且方差相对适中}:  mode + HC 约 101.53\%,标准差 64.62; Raha-Baran + HC 约 90.03\%,标准差 34.80,说明层次聚类在多场景下保持稳定,Raha-Baran 在处理语义或规则错误(如 \textit{hospital, rayyan})时尤为有效。
    \item \textbf{其余组合多集中在 30\%--70\% 区间}:如 Raha-Baran + OPTICS 均值仅 16.75\%,在当前参数设置与数据特征下并未取得良好协同,凸显密度聚类方法在高维或高噪声环境中对参数匹配的依赖度。
\end{itemize}
总体而言,DBSCAN/OPTICS 虽具潜力但波动大,HC 与合适的清洗策略(Mode 或 Raha-Baran)通常兼具更高均值和更低方差。

\subparagraph{低错误率情形下的表现}
在图~\ref{fig:alg_comb_metrics_low} 中仅保留错误率低于 25\% 的数据,用以检验数据质量较佳时的聚类性能及稳定性。结果表明:
\begin{itemize}
    \item \textbf{Mode + HC 均值约 100.14\%},但标准差 74.20:在低错误率场景下,简单填补能够较好保留数据分布,HC 则通过合适的层次切分实现高分。然而,在高维或噪声分布不均的情况下,其表现仍可能出现较大波动。
    \item \textbf{Raha-Baran + HC 稳定性更佳}:均值约 91.04\%,标准差降至 20.04,显示对具备语义或知识库错误的数据集(如 \textit{hospital, rayyan})而言,精细化修复能显著减少噪声干扰,HC 则在较低错误率下持续逼近或略超基准。
    \item \textbf{其余组合受益有限}:如 mode + GMM、mode + KMeans 等平均分多介于 50\%--60\%,虽较全局时略有提升,但在面向 11--20 个特征的中等维度数据时仍不具明显优势。
\end{itemize}

\subsubsection{综合分析与实践建议}
结合上述实验结果和多种清洗与聚类算法的表现,针对不同规模、特征类型及错误率水平的数据集,提出以下建议来指导清洗与聚类策略的选择:

\begin{itemize}
    \item \textbf{小至中规模数据集、高维/多元特征且可能存在语义错误}:
    优先选择 Raha-Baran + HC 组合。Raha-Baran 的上下文修复能力能够有效处理复杂语义错误,而 HC(层次聚类)的分层聚类特性适合高维、多特征数据。若错误率较高,需进一步结合更先进的修复策略,或通过特征选择与降维手段减少特征复杂性。

    \item \textbf{中至大规模数据集、以数值型特征为主、对内部指标最优化有较强需求}:
    考虑使用 mode + HC 或 mode + DBSCAN。简单填补策略(Mode)在数值型数据上的效率较高,而 HC 和 DBSCAN 均能对中大规模数据集进行合理划分,但需谨慎对待 mode + DBSCAN 的高方差风险。若应用场景要求聚类结果与基准结构更贴近且解释性更强,应该选择更稳健的组合(如 mode + HC 或 mode + KMeans)。

    \item \textbf{强噪声分布、不规则簇形状的数据集}:
    可使用 DBSCAN 或 OPTICS 等密度聚类方法。这些方法在处理不规则簇形状和高噪声数据时具有明显优势。密度聚类方法对超参数(如 $\varepsilon$ 和 $\xi$)敏感,因此需要针对不同错误率和分布特征进行精细化超参数调优,以避免出现极端分割(过多单点簇)或过度聚合的情况。自动调优框架(如 Optuna)在此类场景中尤为重要。
\end{itemize}

这些建议为后续应用中选择清洗-聚类方法提供了可行的依据,也为在高噪声及多样化数据环境中平衡聚类质量与可解释性指出了明确方向。

\subsection{自动化聚类模型评估实验}
\label{sec:automl_exp}

为进一步检验多标签分类器在实际应用中的有效性,我们基于第~\ref{sec:large_scale_exp} 节的大规模对比实验,结合训练与测试阶段的聚类结果,综合考量\textbf{损失率} \(\eta(D)\) 与\textbf{综合加速比} \(\mathcal{A}(D)\) 两个核心指标。
为排除极端高分导致的偏差,我们剔除了训练或测试部分得分超过 3.0 的记录,并根据数据来源将数据进行分组统计。表~\ref{tab:autoML_res_new} 展示了各组数据在训练与测试阶段的均分,以及各自使用全量搜索与自动化搜索的平均耗时,并给出了对应的损失率和加速比。

\begin{table}[htbp]
\centering
\small
\setlength{\tabcolsep}{6pt}
\renewcommand{\arraystretch}{1.1}
\begin{tabular}{lcccccc}
\toprule
\textbf{数据集} & \textbf{训练均分} & \textbf{测试均分} & \textbf{全量耗时 (s)} & \textbf{自动化耗时 (s)} & \textbf{\(\eta(D)\)} & \textbf{\(\mathcal{A}(D)\)} \\
\midrule
Flights  & 1.640  & 1.678  & 9,133  & 3,061  & -3.9  & 3.73 \\
Hospital & 1.194  & 1.234  & 8,071  & 6,569  & -10.35 & 1.34 \\
Beers    & 1.221  & 1.399  & 10,575 & 3,591  & -24.78 & 7.10 \\
Rayyan   & 1.537  & 1.740  & 5,916  & 2,170  & -20.47 & 3.90 \\
\midrule
\textbf{整体平均} & \textbf{1.391}  & \textbf{1.508}  & \textbf{8,403}  & \textbf{3,868}  & \textbf{-15.20}  & \textbf{4.02} \\
\bottomrule
\end{tabular}
\caption{不同数据集下自动化管线模型的测试结果}
\label{tab:autoML_res_new}
\end{table}

从表中可以看出,多数数据集的测试均分与训练均分持平或略有上升(例如 \textit{Beers} 与 \textit{Rayyan} 数据集),表明自动化管线在有效缩小搜索空间的同时,依旧能够挖掘到与原有训练基线相当或更高得分的聚类方案。若从损失率角度分析,这一“略有提升”对应负值(如 -3.9\% 至 -24.78\%),即自动化模型选择的子空间方案在测试中未出现显著下降,甚至在部分实验环境下获得更佳表现。

在搜索效率方面,\textit{Beers} 数据集的加速比最高,可达 7.10 倍,说明对于规模较大的数据,自动化管线能极大减少冗余组合的评估,显著降低计算代价。相较之下,\textit{Hospital} 数据集的加速比仅为 1.34 倍,推测其特征分布或先验知识与自动化筛选策略的吻合度较低,导致搜索空间的压缩幅度有限。然而,损失率仍维持在 -10.35\% 附近,表明即使加速效果不算突出,聚类质量并未受到明显损害。

总体而言,所有数据集平均加速比约为 4.02 倍,损失率约为 -15.20\%。这一结果不仅验证了自动化管线对不同类型数据(如数值型、混合型、高维数据)的适应性,也表明其在减少评估耗时的同时,能够保持甚至稍微提升聚类表现。后续研究可在更细粒度上探索各数据集的特征分布与聚类需求之间的交互机制,进一步优化多标签学习的参数设置与筛选策略。

%---------------------------------
% 第六章:结论
%---------------------------------

\section{结论}
\label{sec:conclusion}

本文提出了一种面向数据质量的自动化清洗-聚类优化方法,通过协同优化框架整合数据清洗策略与聚类算法,并利用自动化优化管线缩小搜索空间,以提升聚类效率和质量。研究的主要结论如下:

\begin{enumerate}
    \item \textbf{清洗策略与聚类算法的协同优化是提高聚类质量的关键}。  
    不同清洗-聚类组合在不同数据特征下的适配性差异显著,其中 Raha-Baran + HC 适用于高维、多特征数据,而 mode + DBSCAN 在低维数值数据上可能导致极端分割。

    \item \textbf{自动化管线有效减少搜索开销,同时保持较高聚类质量}。  
    通过多标签学习建模“数据特征—优选方案子空间”的映射,该方法在聚类质量仅有 -15.20\% 平均损失的情况下,实现了平均 4.02 倍的加速,部分数据集甚至在自动化搜索下获得更优结果。

    \item \textbf{数据特征(如错误率、缺失率、噪声水平)直接影响最优策略的选择}。  
    在高错误率场景下,模式填充(mode)易导致偏差,而 Raha-Baran 在语义受限数据(如医疗、文献分析)中的适配性较优。此外,密度聚类(DBSCAN, OPTICS)对超参数敏感度较高,需要更精细的调优策略。
\end{enumerate}

\paragraph{未来工作}  
本研究为数据清洗与聚类优化的协同优化提供了理论支持和实践验证,同时为自动化机器学习在无监督场景下的应用拓展了新方向。后续研究可进一步优化以下方面:

\begin{itemize}
    \item \textbf{数据驱动的自适应清洗策略}:结合知识图谱、深度学习等方法,提升对复杂数据缺陷(如跨属性错误)的识别与修复能力。
    \item \textbf{增量式聚类优化}:针对流式数据,开发动态管线调整机制,使聚类方法能够适应数据分布的变化。
    \item \textbf{超参数智能调优}:结合贝叶斯优化、遗传算法等方法,提高密度聚类的稳定性,并增强模型可解释性。
\end{itemize}

综上,本文研究表明,清洗-聚类协同优化不仅能够提升数据质量对聚类效果的影响控制能力,还能通过自动化优化方法提升搜索效率,为高噪声、大规模数据环境下的聚类任务提供了可扩展、稳健的解决方案。

%---------------------------------
% 参考文献(可选)
%---------------------------------
\bibliographystyle{IEEEtran}
\bibliography{references}

\end{document}
